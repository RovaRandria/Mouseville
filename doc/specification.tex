\section{Sp�cification}
\label{sec:specification}

\paragraph{} Nous avons pr�sent� l'objectif du projet dans la section \ref{sec:introduction}. Dans cette section, nous pr�sentons la sp�cification de notre logiciel r�alis�. Ceci correspond principalement au cahier des charges.

\subsection{Fonctionnalit�s}

\subsubsection{L'environnement} 
	\paragraph{} La simulation se d�roule en tour par tour, dans une grille de X*Y cases, g�n�r�e al�atoirement. Cette grille est compos�e de plusieurs obstacles, et de diff�rentes sources de nourriture, dont la quantit� peut varier. \\
	La grille peut �tre cr�e et modifi�e par l'utilisateur.
	L'utilisateur doit avoir un certain contr�le sur le d�roulement de la simulation.


\subsubsection{Les souris} 
	\paragraph{} L'objectif des souris est de survivre, de ne pas mourir de faim.
Elles se d�placent sur la grille case par case, � la recherche de source de nourriture.
Les souris ont la possibilit� de communiquer entre elles, et s'�changer des informations. \\
Une souris est aussi capable de se dupliquer, et ainsi cr�er une souris totalement identique, tout comme deux souris de sexes oppos�s peuvent donner naissance � plusieurs souris enfants, ayant des caract�ristiques diff�rentes.

	\paragraph{} Chaque souris poss�de une certaine m�moire, permettant de retenir chaque information vue, et de les consid�rer dans ses raisonnements
Des souris � sp�ciales � ont aussi �t� impl�ment�es. Celles-ci ont des caract�ristiques et des comportements bien particuliers, permettant de faire varier la simulation.

\label{sec:spec2}
