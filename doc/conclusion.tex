 \section{Bilan}
\label{sec:bilan}

\subsection{R�ponse au sujet}
	\paragraph{} Nous avons r�alis� une simulation de soci�t� de souris, c'est � dire une repr�sentation du comportement de souris vivant en communaut�, en prenant en compte les besoins de nourriture et reproduction. Les souris survivent en explorant et en gardant en m�moire les sources de nourriture visit�es ou en communicant avec d'autres souris coop�ratives. Une fois bien nourries, les souris peuvent assouvir leur second besoin, celui de se reproduire ( en plus la duplication). Comme dans toutes soci�t�s, nous avons mis en place diff�rents niveaux de confiance et de fiabilit�. De plus, le comportement d'une souris �voluent en fonction de son environnement et des caract�res de leur entourage. Ainsi, une souris  peut devenir tr�s m�fiante avec celles qui ont tent� de la duper.

\subsection{Apports du projet}
	\paragraph{} Ce projet nous a donc beaucoup apport�, notamment au niveau des notions de programmation orient�e objet et de Java; mais surtout sur le plan gestion de projet. Ainsi durant nos s�ances de travaux dirig�es de g�nie logiciel projet, nous avons d�couvert progressivement les diff�rentes �tapes de conception d'un projet et son organisation.
Nous avons d� aussi g�rer un calendrier de t�ches et la synchronisation du travail de groupe, ce qui ne f�t pas une t�che facile..
