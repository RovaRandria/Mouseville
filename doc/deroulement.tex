\section{D�roulement du projet}
\label{sec:deroulement}
 

\subsection{R�partition des t�ches}

	\subsubsection{Zinedine}
		\begin{itemize}
			\item Raisonnement des souris
			\item Gestion de la nourriture
			\item M�moire des souris
		\end{itemize}
	\subsubsection{Matthieu}
		\begin{itemize}
			\item G�n�ration structur�e de la grille
			\item IHM : Grille, Generation, Panneau de controle, Informations cases/souris
			\item Visuel des Souris
			\item Contr�le de la simulation
		\end{itemize}
	\subsubsection{Rova}
		\begin{itemize}
			\item IHM : Optimisation, Layout, M�moire/Vision des souris, Statistiques
			\item Calcul des Statistiques
		\end{itemize}
	\subsection{Synchronisation du travail (SVN)}
		\paragraph{} Nous avons utilis� le logiciel Apache Subversion afin de synchroniser nos travaux. Ce logiciel nous a permis de travailler en �quipe de mani�re simple et rigoureuse. Nous avons effectu� des commits au rythme de un par semaine en moyenne. Les commits furent tr�s rapproch�s dans le temps au d�but du projet afin que tout le monde ait acc�s aux classes de donn�es. � l'inverse, � la fin du projet, les modifications apport�es par les membres de l'�quipe �tant mineures, les commits �taient beaucoup plus �loign�s dans le temps. Nous avions fix� une certaine norme � respecter par rapport aux messages de commit : les messages doivent �tre r�dig�s int�gralement en anglais.  Les modifications doivent �tre sp�cifi�es sous forme de listes pr�c�d�es du type de modification (ajout, modification, am�lioration).

	\subsection{Calendrier}
		
		\fig{images/calendrier.png}{10cm}{13cm}{Calendrier}{cal}

	\paragraph{} Le calendrier (voir figure \ref{fig:cal}) liste chronoligiquement, sans d�tail, les diff�rentes t�ches (regroup�es par semaine) que nous avons ex�cut� au cours de ces mois de projet.

